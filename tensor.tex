\documentclass{article}
\usepackage{amsthm}
\usepackage{amsfonts}
\usepackage{amsmath}
\usepackage{tikz-cd}
\usepackage{tikz}
\usetikzlibrary{matrix,arrows,decorations.pathmorphing}

\theoremstyle{plain}
  \newtheorem{lemma}{Lemma}[section]
  \newtheorem{prop}{Proposition}[section]
  \newtheorem{thm}{Theorem}[section]
  \newtheorem{cor}{Corollary}[section]

\theoremstyle{definition}
  \newtheorem{defn}{Definition}[section]

\theoremstyle{remark}
  \newtheorem{remark}{Remark}[section]
  \newtheorem{eg}{Example}[section]
  \newtheorem{ceg}{Counterexample}[section]

\begin{document}
  Abstract Algebra presentation on Tensor Product, Eric Fay\\ Rutgers University. \\[1em]
  For a digital copy for to see source visit
  github.com/PersonaErazed/tensors
  \subsection{A brief reminder of what Modules are}
  \begin{defn}
    Module.
    Let $R$ be a ring. A set $M$ with a binary operation 
    $+:M\times M\to M$ and a map $.:R\times M\to M$ is a 
    \emph{left $R$-module} iff
    \begin{enumerate}
      \item $(M,+)$ is an Abelian group
      \item $.:R\times M\to M$ defines a ``ring action of $R$ on $M$''
        (scalar multiplication) i.e.
        for all $r,s\in R$ and for all $m,n\in M$ that
        \begin{enumerate}
          \item $(r+s).m=r.m+s.m$
          \item $(r\cdot s).m=r.(s.m)$
          \item $r.(m+n)=r.m+r.n$
        \end{enumerate}
      \item if $R$ is a ring with identity then $1.m=m$ for all $m\in M$
    \end{enumerate}
  \end{defn}
  \begin{eg}
    Any ring \[r.s = r\cdot s\]
%    Any ring $R$ is a $R$-module since $(R,+)$ is an Abelian group
%    and $\cdot:R\times R\to R$ is an associative binary operation which 
%    distributes over addition by definition.
  \end{eg}
  \begin{eg}
    Abelian groups are $\mathbb{Z}$-modules
    \[n.a=\left\{\begin{array}{llr} 
        a+\cdots+a &(n-1) \text{ addition operations}& :n>0 \\
        0 && :n=0 \\
        -a+\cdots+ -a &(-n-1) \text{ addition operations} &:n<0
      \end{array}\right.\]
  \end{eg}
  \begin{eg}
%    Vector spaces over a field $\mathbb{F}$ are $\mathbb{F}$-modules.
    Vector spaces are $R$-modules where $R$ is a field.
  \end{eg}
  \begin{eg}
    Collection of maps from a set $X$ to a $R$-module $M$  
    \[M^X=\{ f:X\to M \}\]
%    $(f+g)(x)=f(x)+g(x)$ and $(r.f)(x)=r.f(x)$
    where for all $f,g\in M^X$ and for all $x\in X$ and for all $r\in R$
    \[(f+g)(x)=f(x)+g(x)\qquad(r.f)(x)=r.f(x)\]
  \end{eg}
  \begin{eg}
    $X$ is a finite set e.g. $X=[n]:=\{1,2,\cdots,n\}$ for some 
    natural number $n$.
    \\[1em]
    Then $M^X=M^{[n]}$ is a $R$-module for some 
    $M$ $R$-module i.e. direct product of $R$-modules 
    \[M^n=M\times \cdots\times M\]
  \end{eg}
  \begin{eg}
    Smooth functions on a real smooth $n$-manifold $X$ such as $\mathbb{R}^n$
    \[\mathcal{C}^\infty(X)=\{f:X\to\mathbb{R} | f\text{ is a smooth map
    }\}\subset\mathbb{R}^X\]
    smooth map\footnote{
      $f$ is a smooth map $\iff$ for all smooth charts $(\phi,U)$ that 
      the following is a smooth function
      \[\phi^{-1}\circ f:\phi(U\cap V)\subset\mathbb{R}^n\to\mathbb{R}\] 
    } 
  \end{eg}
  \begin{eg}
    The smooth vector fields on a manifold $X$ is a 
    $\mathcal{C}^\infty(X)$-module
    \[ \Gamma(TX)=\{v:X\to TX |\text{ smooth and }\pi\circ v=\text{id}_X\}\]
    where $\pi:TX\to X$ is the projection from the tangent bundle
    $TX=\amalg_{x\in X}T_xX$, the disjoint union of the tangent spaces, onto X.
    \[ (f.v)(x)=f(x).v_x \]
  \end{eg}
\section{Tensor Product}
\subsection{Special Case: Extension of Scalar Ring}
%%%%%%%%%% Dummit Foote %%%%%
  Let us consider the motivating problem presented by Dummit and Foote. Given
  a $R$-module such that the ring $R$ is a subring of another ring $S$, what is the most
  general $S$-module which the given $R$-module can be embedded? i.e. when can
  a $R$-module to extended to a $S$-module? This is know as extension of 
  scalars.

  Let us call our $R$-module to be extended $N$.
  Recall the key elements of the definition for a $\mathcal{R}$-module, a Abelian group
  structure and an action of the ring $\mathcal{R}$ on the Abelian group.
  It is therefore natural to consider the free Abelian group $F(S\times N)$
  as a candidate Abelian group to define a $S$ ring action on and which $N$
  can be embedded.

  \begin{defn}
     For any ring $\mathcal{R}$, a $\mathcal{R}$-module $F$ is \textbf{free} iff
     for some set $X$, there exists unique nonzero elements 
     $r_1,\ldots,r_k\in\mathcal{R}$ and unique elements $x_1,\ldots,x_k\in X$
     for each element $f\in F$, such that $f=r_1.x_1+\cdots+r_k.x_k$
  \end{defn}
  \begin{eg}
    Any ring $\mathcal{R}$ and any set $X$ let 
    \[F(X)=\{f:X\to \mathcal{R}\;|\;X\setminus f^{-1}[0] \text{ finite}\}\subset \mathcal{R}^X\]
    i.e. maps $X\to \mathcal{R}$ with finte support where support is defined as supp$(f):=\{x\in X:f(x)\neq 0\}$.
    The $\mathcal{R}$-module structure is defined componentwise.
\\[1em]
    To show $F(X)$ is free
    define the injection $\iota:X\to F(X):x\mapsto \delta_x$ such that
    \[\delta_x(y)=\left\{\begin{array}{lr}1&y=x\\0&y\neq x\end{array}\right.\]
    It is easy to see
    for any $f\in F(X)$ there exists unique nonzero elements $r_1,\ldots,r_k\in\mathcal{R}$
    and unqiue elements $x_1,\ldots,x_k\in X$ such that \[f=r_1.\delta_{x_1}+\cdots+r_k.\delta_{x_k}\]
  \end{eg}
  \begin{remark}
    $\mathcal{R}=\mathbb{Z}$ then $F$ a free $\mathbb{Z}$-module is called a
    free Abelian group.
   \\ 
  \end{remark}

  With $F(S\times N)$ as our Abelian group let us define a ring action from $S$
  as for any $f\in F(S\times N)$ and $s\in S$
  \begin{align*}
    s.f:&=s.(s_1,n_1)+\cdots+s.(s_k,n_k) \\
    &=(s\cdot s_1,n_1)+\cdots+(s\cdot s_k,n_k)
  \end{align*}
  Now considering the details of the definition of a general $\mathcal{R}$-module $M$ the
  ring action must preserve both the ring's and Abelian group's structures. i.e.
  \begin{align*}
    &(r_1+r_2).m = r_1.m + r_2.m 
    &(r_1\cdot r_2).m = r_1.(r_2.m)\\
    &r.(m_1+m_2)=r.m_1 + r.m_2
  \end{align*}
  for any $r,r_1,r_2\in\mathcal{R}$ and any $m,m_1,m_2\in M$.

  This leads us to define the equivalence relation $\sim$ on $F(S\times N)$ as follows
  \begin{align*}
    &(s_1+s_2,n)\sim(s_1,n) + (s_2,n)
    &(s\cdot r,n)\sim(s,r.n) \\
    &(s,n_1+n_2)\sim(s,n_1)+(s,n_2)
  \end{align*}
  for any $s,s_1,s_2\in S$, any $r\in R$, and any $n,n_1,n_2\in N$.
  Note the subtle difference in the compatibility relation this is the case
  since $N$ has a $R$-module structure.

  Since equivalence relations give us a normal subgroup i.e. $a\sim b \iff a-b\in H$.
  Let us define the $S$-module we have constructed as 
  \[S\otimes_R N :=F(S\times N)/\sim \]
  let us call the cosets, tensors, the following is an example
  of a simple tensor
  \[s\otimes n:=\{f\in F(S\times N):f\sim (s,n)\}\]
  The binary Abelian operation + is defined via the natural surjective
  group homomorphism $\pi:F(S\times N)\to F(S\times N)/\sim$
  \begin{align*}
    (s\otimes n)+(s'\otimes n') &= \pi((s,n))+\pi((s',n')) \\
     &= \pi((s,n)+(s',n')) \\
     &= \{f\in F(S\times N):f\sim (s,n)+(s',n')\}
  \end{align*}
  This yeilds the following relations
  \begin{align*}
    &(s_1+s_2)\otimes n = s_1\otimes n + s_2\otimes n
    &(s\cdot r)\otimes n = s.(r.n)\\
    &s\otimes(n_1+n_2)=s\otimes n_1+ s\otimes n_2
  \end{align*}
  Let us define the $S$ ring action as
  \[s_\circ\left(\sum_{i\in[n]}s_i\otimes n_i\right)
   = \sum_{i\in[n]}(s\cdot s_i)\otimes n_i\]
  Exercise check if this is well defined, note these are cosets and
  check for simple tensors $s\otimes n$ that this makes $S\otimes_R N$ a
  $S$-module.
 
%  To review we considered the product of two $R$-modules right and left 
%  respectively i.e. $S$ is a left $S$-module as well as a right module-$R$.
%  We extended the

  The following theorem shows this is the most general $S$-module to embedded
  the $R$-module $N$ into. First we need a lemma the Universal property of 
  free modules.
  \begin{lemma}
    Given any set $X$, any left $R$-module $M$, and any map $\psi:X\to M$.
    There exists a unique $R$-module homomorphism from the free 
    $R$-module generated by $X$ to $M$    
    \[\Psi:F(X)\to M \qquad\text{s.t.}\quad \psi(x)=\Psi(x) \forall x\in X \]
  \end{lemma}
  The proof follows since $X$ is a generating set of $F(X)$ and each element
  of $F(X)$ has a unique representation of the formal 
  form $r_1x_1+\cdots+r_kx_k$ see example 1.1

  The map is \[\Psi:\sum_{i\in[k]}r_ix_i\mapsto \sum_{i\in[k]}r_i\psi(x_i)\]
  \begin{thm}
    Let $R \subset S$  be a subring, $N$ be a $R$-module, and 
    $\iota:N\to S\otimes_R N$ be the $R$-module homomorphism 
    $n \mapsto 1\otimes n$.\\[1em]
    For any left $S$-module $L$ and $R$-module homomorphism
    $\phi:N\to L$ there exists a unique $S$-module homomorphism
    $\Phi:S\otimes_R N\to L$ such that $\phi=\Phi\circ\iota$
\begin{center}\begin{tikzcd}
N \arrow{r}{\iota}\arrow{rd}{}[swap]{\phi}
&S\otimes_RN \arrow[dotted]{d}{\Phi}[swap]{!}\\
&L
\end{tikzcd}\end{center}
  \end{thm}
  \begin{proof}
    Since $L$ is a left $S$-module let us define the map 
    \[\psi:S\times N\to L:(s,n)\mapsto s.\phi(n)\]  
    By the universal property of free modules, see Lemma 1.1, there exist 
    unqiue $\mathbb{Z}$-module homomorphism from the free Abelian group to the
    left $S$-module $L$ 
    \[  \Psi:F(S\times N)\to L \]
    Specifically
    \begin{align*}
      \Psi\left( \sum_{i\in[k]}a_i(s_i,n_i) \right)
      &= \sum_{i\in[k]}a_i\psi((s_i,n_i)) \\
      &= \sum_{i\in[k]}\sum_{j=1}^{a_i}\psi((s_i,n_i))\\
      &= \sum_{i\in[k]}\sum_{j=1}^{a_i}s_i.\phi(n_i)
    \end{align*}
    Since $\phi$ is an $R$-module homomorphism i.e.
    \[\phi(x+y)=\phi(x)+\phi(y)\qquad\text{and}\qquad\phi(r.x)=r.\phi(x)\]
    Then the generators for the subgroup from the equivalence relation $\sim$ discussed
    previously map to zero in $L$.
    e.g. $(s_1+s_2,n)\sim(s_1,n)+(s_2,n)$ gives a generatoring element
    $(s_1+s_2,n)-(s_1,n)-(s_2,n)$ which maps to
    \begin{align*}
      \Psi((s_1+s_2,n)-(s_1,n)-(s_2,n))
      &=\psi((s_1+s_2,n))-\psi((s_1,n))-\psi((s_2,n))\\
      &=(s_1+s_2).\phi(n)-s_1.\phi(n)-s_2.\phi(n)\\
      &=s_1.\phi(n)+s_2.\phi(n)-s_1.\phi(n)-s_2.\phi(n)\\
      &=0
    \end{align*}
    Similarly for $(s\cdot r,n)\sim(s,r\cdot n)$ and $(s,n_1+n_2)\sim(s,n_1)+(s,n_2)$
    Therefore there exists a well defined $\mathbb{Z}$-module homomorphism
    \[ \Phi:F(S\times N)/\sim=S\otimes_RN\to L \]
    where $\Phi(\sum s_i\otimes n_i)=\sum s_i.\phi(n_i)\in L$. Since $L$ is a left
    $S$-module
    \begin{align*}
      s.\Phi\left(\sum s_i\otimes n_i\right) &= s.\left(\sum s_i.\phi(n_i)\right) \\
      &= \sum s.(s_i.\phi(n_i)) \\
      &= \sum (s\cdot s_i).\phi(n_i) \\
      &= \Phi\left(\sum(s\cdot s_i)\otimes n_i\right)\\
      &= \Phi\left(\sum s.(s_i\otimes n_i)\right)
    \end{align*}
    From the above calculation we see $\Phi$ is a $S$-module homomorphism.
    Note any $S$-module homomorphism is uniquely determined by the values on
    elements of the generating set, e.g. $1\otimes n$ and since 
    $\Phi(1\otimes n)=\phi(n)$ then $\Phi$ is uniquely determined by $\phi$.
  \end{proof}
  \begin{remark}
    The converse of this theorem is true. If $\Phi:S\otimes_RN\to L$ is a 
    $S$-module homomorphism then $\phi=\Phi\circ\iota:N\to L$ is a
    $R$-module homomorphism.
  \end{remark}
  \begin{cor}
    $\frac{N}{ker\iota}$ is the unique largest quotient of $N$ that can be 
    embedded in any $S$-module.
  \end{cor}
  \subsection{General Tensor}
  Let us consider only the necessary properties from the construction
  of the most general module for a given scalar extension
  and apply it to a construct
  of a space in which the product of elements from two spaces can have
  a group structure and given sufficient conditions have a module structure.

  Observing the relations used in the extenision of scalars when note the 
  requirements of the constituent spaces of the construct needs an abelian
  group structure. By the relation $(x.r,y)\sim(x,r.y)$ the constituent
  spaces need a right and left ring action respectivaly. This leads us
  to the construction of the abelian group known as the tensor product.
  \begin{defn}\textbf{Tensor Product}\\
    Let $M$ be any right module-$R$ and $N$ any left $R$-module. Consider the
    equivalance relation $\sim$ on $F(M\times N)$ free Abelian group generated from
    $M\times N$ where
  \begin{align*}
    &(m_1+m_2,n)\sim(m_1,n) + (m_2,n)
    &(m.r,n)\sim(s,r.n) \\
    &(m,n_1+n_2)\sim(m,n_1)+(m,n_2)
  \end{align*}
  The abelian group $M\otimes_RN=F(M\times N)/\sim$ is called the \textbf{tensor product of $M$ and $N$ over $R$}
  \end{defn}
  \begin{remark}
    In the above definition of tensor product if the right module-$R$ was a
    bimodule or has a left ring action by another ring $S$ then the tensor
    product has a natural module structure. Defining it on a simple tensor
    as $s.(m\otimes n):= (s.m)\otimes n$

    If $R$ is a commutative ring we define $m.r:=r.m$ for a right module-$R$
    $M$ which can be shown that $M$ is a $R$-bimodule.
    Note we can see $R$ must be commutative since
    \[(r\cdot r').m=m.(r\cdot r')=(m.r).r'=r'.(r.m)=(r'\cdot r).m\]
  \end{remark}
  In the special case in extension of scalars we considered $R$-module 
  homomorphisms to any $S$-module where $S$ contained $R$. In the general
  case we consider a more general type of map, one which still sends the
  generators of kernel of $\pi:F(M\times N)\to M\otimes_RN$ e.g. $(m_1+m_2,n)-(m_1,n)-(m_2,n)$.
  \begin{defn}\textbf{R-balanced map}\\
    Let $M$ be any right module-$R$, $N$ any left $R$-module, and $L$ any 
    Abelian group.\\[1em]
    $\phi:M\times N\to L$ is $R$-balanced iff
    \begin{align*}
      &\phi(m_1+m_2,n)=\phi(m_1,n)+\phi(m_2,n)
      &\phi(m,r.n)=\phi(m.r,n) \\
      &\phi(m,n_1+n_2)=\phi(m,n_1)+\phi(m,n_2)
    \end{align*}
  \end{defn}
  \begin{thm}\textbf{Universal Property of Tensor Product}\\
    Let $R$ be any ring with 1, $M$ right module-$R$, $N$ left
    $R$-module, and $L$ any Abelian group. \\[1em]
    If $\phi:M\times N\to L$ is a $R$-balanced map then there exists a 
    unique group homomorphism $\Phi:M\otimes_RN\to L$ such that $\phi=\Phi\circ\iota$.\\[1em]
    If $\Phi:M\otimes_RN\to L$ any group homomorphism then 
    $\phi=\Phi\circ\iota:M\times N\to L$ is a $R$-balanced map.
  \end{thm}
  Since the product of sets is associative it seems likely that the tensor
  product is also associative. The next theorem shows this is the case.
  \begin{thm}\textbf{Associativity of the Tensor Product}\\
    Let $M$ be any right $S$-module-$R$, $N$ any $R$-module-$T$, and $L$
    any left $T$-module.
    \\[1em]
    There exists a unique $S$-module isomorphism
   \[ (M\otimes_RN)\otimes_TL\cong M\otimes_R(N\otimes_T L) \]
   $m\otimes(n\otimes l)\mapsto (m\otimes n)\otimes l$
  \end{thm}
  \begin{proof}
    Note that $M\otimes_RN$ is $S$-module-$T$ and $N\otimes_TL$ left $R$-module.
    For each $l\in L$ define $\phi_l^L:M\times N\to M\otimes_R(N\otimes_TL)$
    \[\phi_l^L(m,n) = m\otimes (n\otimes l)\]
    \begin{align*}
      \phi_l^L(m.r,n) &= m.r\otimes(n\otimes l) \\
      &= m\otimes r.(n\otimes l) \\
      &= m\otimes (r.n\otimes l) \\
      &= \phi_l^L(m,r.n)
    \end{align*}
    The above calculation shows $\phi_l^L$ is $R$-balanced. By the universal 
    property of tensor product there exist a unique homomorphism
    \[ \Phi_l^L:M\otimes_RN\to M\otimes_R(N\otimes_TL) \]
    and therefore the following map is well defined
    \begin{align*}
      \Phi^L&:(M\otimes_RN)\times L\to M\otimes_R(N\otimes_T L)\\
      &:(m\otimes n,l)\mapsto \Phi_l^L(m\otimes n)
    \end{align*}
    A simple calculatin shows that $\Phi^L$ is $R$-balanced and therefore by
    the universal property of tensor product there exist a unique homomorphism
    $\Phi:(M\otimes_R N)\otimes_TL\to M\otimes_R(N\otimes_TL)$. Repeat with
    $\psi_m^M$ maps to get the inverse homomorphism.

    To complete the proof note that $S$ action is compatible with the $T$ action
    \[s.((m\otimes n).t)=s.(m\otimes n.t)=s.m\otimes n.t = (s.m\otimes n).t=(s.(m\otimes n)).t \]
    so $M\otimes_RN$ is a $S$-module-$T$ making $(M\otimes_RN)\otimes_TL$ a $S$-module.
    By the isomorphism constructed above $\Phi(s.(m\otimes n)\otimes l)=\Phi((s.m\otimes n)\otimes l)=s.m\otimes(n\otimes l)=s.(m\otimes(n\otimes l))$
  \end{proof}
  Most spaces of interest for which we construct a tensor product have
  a have action by a commutative ring and therefore are bimodules. Such
  as vector spaces. In this case the $R$-balanced maps are multilinear maps.
  \begin{defn}\textbf{Multilinear Map}\\
    Consider $R$ commutative ring and $M_1,\ldots,M_n$ and $L$ be 
    $R$-modules. Then a map $\phi:M_1\times\cdots\times M_n\to L$
    is a multilinear map if and only if for each component $i\in[n]$
    the map where the other entries are fixed is a $R$-module homomorphism.
    \[\phi(m_1,\ldots,m_{i-1},r.m_i+r'.m_i',m_{i+1}\ldots,m_n)=\]
    \[r.\phi(m_1,\ldots,m_{i-1},m_i,m_{i+1}\ldots,m_n)+
    r'.\phi(m_1,\ldots,m_{i-1},m_i',m_{i+1}\ldots,m_n)\]
  \end{defn}
  \begin{cor}
    Let $R$ be a commutative ring and $M_1,\cdots,M_n$ be $R$-modules.
    Let $M_1\otimes\cdots\otimes M_n$ be any bracketing of the tensor 
    product of the aforementioned modules. Let
    \begin{align*}
      \iota&:M_1\times\cdots\times M_n\to M_1\otimes\cdots\otimes M_n \\
           &:(m_1,\ldots,m_n)\mapsto m_1\otimes\cdots\otimes m_n
    \end{align*}
    For any $R$-module $L$ there is a bijection
    between multilinear maps and $R$-module homomorphism
    \begin{align*}
      \left\{\underset{\text{ mulitlinear}}{\phi:M_1\times\cdots\times M_n\to L}\right\}
      \left\{\underset{\text{ $R$-module homomorphism}}{\Phi:M_1\otimes\cdots\otimes M_n\to L}\right\}
    \end{align*}
    such that the following diagram commutes
    
\begin{center}\begin{tikzcd}
M_1\times\cdots\times M_n \arrow{r}{\iota}\arrow{rd}{}[swap]{\phi}
&M_1\otimes\cdots\otimes M_n \arrow[dotted]{d}{\Phi}[swap]{!}\\
&L
\end{tikzcd}\end{center}
  \end{cor}
  \begin{thm}\textbf{Tensor product commutes with Direct Sum}\\
    Let $M$,$M'$ be any right modules-$R$ and $N,N'$ be any
    left $R$-modules.\\[1em]
    There exists a unique group isomorphisms
    \[(M\oplus M')\otimes_R N\cong(M\otimes_RN)\oplus(M'\otimes N)\]
    \[ M\otimes_R(N\oplus N')\cong(M\otimes_RN)\oplus(M\otimes_RN') \]
    such that $(m,m')\otimes n \mapsto (m\otimes n,m'\otimes n)$ and
    $m\otimes(n,n')\mapsto(m\otimes n,m\otimes n')$ respectively.
    \\[1em]
    And if $M,M'$ $S$-module-$R$ then there exists a $S$-module isomorphism.
  \end{thm}
  \begin{proof}
    Define the map 
    \begin{align*}
      \phi&:(M\oplus M')\times N\to(M\otimes N)\oplus(M'\otimes N)\\
            &:((m,m'),n)\mapsto (m\otimes n,m'\otimes n)
    \end{align*}
    Check if it is well defined and $R$-balanced.
    Then use the universal property of tensor products to get a homomorphism
    \begin{align*}
      \Phi&:(M\oplus M')\otimes N\to(M\otimes N)\oplus(M'\otimes N)\\
            &:(m,m')\otimes n\mapsto (m\otimes n,m'\otimes n)
    \end{align*}
    Now define two maps
    \begin{align*}
      \psi&:M\times N\to(M\oplus M')\otimes N
      &\psi'&:M'\times N\to(M\oplus M')\otimes N\\
          &:(m,n)\mapsto (m,0)\otimes n
          &&:(m',n)\mapsto (0,m')\otimes n
    \end{align*}
    Again check if they are $R$-balanced and use the universal property 
    to get the unique homomorphisms
    \begin{align*}
      \Psi&:M\otimes N\to(M\oplus M')\otimes N
      &\Psi'&:M'\otimes N\to(M\oplus M')\otimes N\\
          &:m\otimes n\mapsto (m,0)\otimes n
          &&:m\otimes n\mapsto (0,m')\otimes n
    \end{align*}
    Next define the map 
    \begin{align*}
      \Psi\oplus\Psi'&:(M\otimes N)\oplus(M'\otimes N)\to(M\oplus M')\otimes N \\
      &:(m\otimes n,m'\otimes n)\mapsto \Psi(m\otimes n)+\Psi(m'\otimes n)
    \end{align*}
    Compute to show $\Phi\circ(\Psi\oplus\Psi')=id_{(M\oplus M')\otimes N}$
    and $(\Psi\oplus\Psi')\circ\Phi=id_{(M\otimes N)\oplus(M'\otimes N)}$
  \end{proof}
\end{document}
